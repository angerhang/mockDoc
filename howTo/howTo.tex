\documentclass{book}

\title{\textbf{How To Write A Minimal LaTeXML Binding}}
\author{Author: Hang Yuan, Jinbo Zhang \\ \\ \\ Supervisor: Michael Kohlhase}

\begin{document}
\maketitle

\tableofcontents

\large\chapter{Introduction}
\paragraph \LaTeX  has been widely used as the word processing tool among scholars, especially 
when one needs to use large quantities of mathematical representations. \LaTeX is also 
good for those who are meticulous about typographical quality of the documents.  However,
\LaTeX \ lacks of conversion tool to XML which Digital Library of Mathematical Functions uses
for delivery. DLMF developed \LaTeX ML , trying to make a new typesetting system that allows
users to be able to focus more on the content, but not the style, by providing extensive ways of 
customizations. In order to achieve this goal, building up the document class binding seems crucial, 
and yet \LaTeX ML seems fairly unfathomable for beginners. We want to make it easier for people who
want to pick up using \LaTeX ML in the future, by going through how to construct a minimal \LaTeX ML 
binding step by step.  \\

\large\paragraph  Again, this document does not cover advanced topics related \LaTeX  ML, and thus if 
you are interested in understanding how this and how that, it is very likely that the \LaTeX ML manual will
serve your needs better. 
 
\chapter{Understanding \LaTeX ML} 
\section{Using LaTeXML}
The first thing we want to talk about here is, what aspects of \LaTeX ML we are going to cover, and the 
workflow of creating your first \LaTeX document class binding. In this particular tutorial, we will use command:\\ \\
\textbf{latexml} for converting \TeX document into *.XML \\ \\
The general command for conversion is \\ \\
\textbf{latexml} {options} --destination=doc.xml doc \\ \\
Or simply you only supply with the \TeX file and the result will be standard output which is totally fine as well,
based on your needs. \\
One quick note here about \LaTeX ML installation, when you think you have finish installing \LaTeX ML, run
a simple conversion command within mockDoc.tex's directory. You should be able to see an XML interpretation of
mockDoc.tex either in a form a standard output or a newly-generated XML file. If you have something that differs from
the expected and have check your \LaTeX ML package multiple times, maybe you have overlooked some prerequisites such as 
libxml2 and libxslt. 
\section{LaTeXML Binding}
\subsection{Minimal LaTeXML Structure}
\subsection{RelaxNG Schema}
\chapter{References}



\end{document}