\documentclass{book}
\usepackage{listings}

\title{\textbf{How To Write A Minimal LaTeXML Binding}}
\author{Author: Hang Yuan, Jinbo Zhang \\ \\ \\ Supervisor: Michael Kohlhase}

\begin{document}
\maketitle

\large\chapter{Introduction}
\paragraph \LaTeX  has been widely used as the word processing tool among scholars, especially 
when one needs to use large quantities of mathematical representations. \LaTeX is also 
good for those who are meticulous about typographical quality of the documents.  However,
\LaTeX \ lacks of conversion tool to XML which Digital Library of Mathematical Functions uses
for delivery. DLMF developed \LaTeX ML , trying to make a new typesetting system that allows
users to be able to focus more on the content, but not the style, by providing extensive ways of 
customizations. In order to achieve this goal, building up the document class binding seems crucial, 
and yet \LaTeX ML seems fairly unfathomable for beginners. We want to make it easier for those who
want to pick up using \LaTeX ML in the future, by going through how to construct a minimal \LaTeX ML 
binding step by step.  \\

\large\paragraph  This document does not cover advanced topics related \LaTeX  ML, and thus if 
you are interested in understanding how this and how that work, very likely the \LaTeX ML manual will
serve your needs better. 
 
\chapter{Understanding \LaTeX ML} 
\section{Using LaTeXML}
The first thing we want to talk about is, what aspects of \LaTeX ML we are going to cover, and then we will move onto the 
workflow of creating your first \LaTeX\ document class binding. In this  tutorial, we  use command:\\ \\
\textbf{latexml} for converting \TeX \ document into *.XML \\ \\
The general command for conversion is \\ \\
\textbf{latexml} {options} --destination=doc.xml doc \\ \\
Or simply you only supply with the \TeX\ file and the result will be standard output which is totally fine as well,
based on your needs. \\
One quick note here about \LaTeX ML installation, when you think you have finished installing \LaTeX ML, run
a simple conversion command within mockDoc.tex's directory. You should be able to see an XML interpretation of
mockDoc.tex either in a form a standard output or a newly-generated XML file. If you have something that differs from
the expected and you have already checked your \LaTeX ML package multiple times, maybe you have overlooked some prerequisites such as 
libxml2 and libxslt. \\ \\
\textbf{Note:}
Now in order to make better use of our document class binding in the future, we need to know how \LaTeX ML 
operates in different stages. The stages are like the following: Digestion , construction, rewriting, math parsing and serialization.
However we are not going to discuss this in detail here. For the interested users, the \LaTeX ML manual is a good source.
\section{LaTeXML Binding}
The conversion from \TeX\  to XML is processed by \LaTeX ML. Basically \LaTeX ML maps the \TeX \ makeup to
the XML markup, more specifically macros, primitives and constructors. That's why you are able to customize the conversion 
between \TeX\  and XML, in three ways, modifying the bindings used by \textbf{latexml}, adding your own bindings that has not been
implemented, and even creating your own \TeX \ style and \LaTeX\  binding which is exactly the goal of this tutorial.
\subsection{Minimal LaTeXML Structure}
Since \LaTeX binding is essentially a Perl module, we need to initialize a binding file by add the followings in the beginning of the document class:
\begin{lstlisting}
package LaTeXML::Package::Pool; //load text package 
use strict; // catch errors and stop when encounter one
use LaTeXML::Package;

//your customization here 

use warnings; // give warnings
1; //make sure Perl work properly 
\end{lstlisting}

We always need to load Tex.pool binding and possibly LaTex.pool as well in the beginning. \LaTeX ML packages are just like the 
style and class files in \TeX\  and they have an extension of \emph{.ltxml}. We load our \LaTeX binding juts like how we load our 
class file in \TeX . In our case we would like to load \emph{doc.cls.ltxml}, and therefore we use \emph{\textbackslash documentclasss\{doc\}},
similarly, if you would like to load a doc type doc.sty.ltxml, you only need to include \emph{\textbackslash use package\{doc\}} in your .tex 
file.

\section{Construction \& Constructors}
As we are interested in the conversion to XML, we need to understand how constructors works. \\ \\
\textbf{DefConstructor} \emph{ (\$prototype, \$replacement, \$options) } \\ \\
\textbf{DefConstructor}(`\ \textbackslash section{}', ``\textless mock:section\textgreater\textless mock:title\textgreater \#1\textless/mock:title\textgreater");
The prototype is the control sequence you have defined in your \emph{.tex} file, and the replacement is what you want it to be built in your XML file. There 
are options that need to be passed. In our doc.cls.ltxml, it is important to include \\ \\
 \emph {beforeDigest =\textless sub \{ AssignValue(inPreamble =\textless 0);\}} \\ \\
 for it makes sure that there spaces between words in the generated XML file. For other similar options, you might not necessarily be able to find them in 
 the manual, albeit, you can go one step further by looking at the other built in bindings such as \emph {article.cls.ltxml} and the \emph {pool} package. \\ \\
 One other error you might encounter when creating a biding from scratch is the constructor for new line. The macro for new line in \TeX is  \emph{\textbackslash\textbackslash}. \\
 It does not necessarily mean you are able to customize  \emph{\textbackslash\textbackslash}, due to some predefinition in the \emph {pool} package, which explains 
 why we change our the conventional  \emph{\textbackslash\textbackslash} in \emph{mockDoc} into \emph{ \textbackslash newline}. The same method can be considered
 as one possible solution for some \emph{malformed} errors.

\section{Document Model}
After customizing how \TeX  is translated into XML. There are three more schema to include: RelaxNGSchema, RegisterNamespace, and Tag.
\subsection{RelaxNGSchema}
The constructors tell \LaTeX ML to add a replacement into XML when it detects a prototype in the \emph{.tex} file. The question is it doesn't tell
where the tag should be closed, and therefore we introduce \\ \\
 \emph{Tag(\$Tag,\$properties)} \\ \\
 meaning whenever there is new paragraph, the last paragraph tag will be closed, before the new paragraph tag can be added. RelaxNGSchema
 tells how the whole document class in constructed, such that when we want to create our own document structure, RelaxNGschema is quite important. \\ \\
 \LaTeX ML is unable to process the compact form which has an extension of .rnc but .rng. It makes your life easier to write in compact form nevetnhelress.
 The trick is to convert your .rnc in to .rng using trang. \\ \\ 
 In a schema file, pretty much you are defining in what elements can be included in different bodies. 
 As you can see in \emph{mockDoc.rnc} file. We need to register our namespace first, which we need to do later in \LaTeX ML file as well.
 The the definition of paragraph tells \LaTeX ML, in a paragraph it has two elements that are title and p. Likewise, section contains a title followed by two possible \\
structures either title or subsection. \\ \\
Finally we come to last part of our \LaTeX ML, namespace. Namespace differentiates our customized macros with others by adding a prefix to our \LaTeX ML constructor patterns. \\ \\
\emph{RegisterNamespace(\$prefix, \$url) \\} 

Congratulations for being able to follow this tutorial to the end. You should have everything you need to create a minimal document \LaTeX ML binding. After the processing the makefile, you should be able to see the generated XML in your current directory! \\

\textbf{Reference} \\
 I. Bruce R. Miller, November 21, 2014 \emph{\LaTeX ML The Manual} \\
 II. 2006 - 2012 Philipp Lehman, \emph{biblatex}
 % III. kwarc.bib

\end{document}