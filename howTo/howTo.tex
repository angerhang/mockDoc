\documentclass{book}
\usepackage{listings}
\usepackage{tcolorbox}
\usepackage{hyperref} 
\title{\textbf{How To Write A Minimal LaTeXML Binding}}
\author{Author: Hang Yuan, Jinbo Zhang \\ \\ \\ Supervisor: Michael Kohlhase}

\lstset{numbers=left, numberstyle=\tiny, stepnumber=1, numbersep=5pt,  breaklines=true, language=Octave}
\begin{document}
\maketitle

\large\chapter{Introduction}
\paragraph \LaTeX  has been widely used as the word processing tool among scholars, especially 
when one needs to use large quantities of mathematical representations. \LaTeX is also 
good for those who are meticulous about typographical quality of the documents.  However,
\LaTeX \ lacks of conversion tool to XML which Digital Library of Mathematical Functions uses
for delivery. DLMF developed \LaTeX ML , trying to make a new typesetting system that allows
users to be able to focus more on the content, but not the style, by providing extensive ways of 
customizations. In order to achieve this goal, building up the document class binding seems crucial, 
and yet \LaTeX ML seems fairly unfathomable for beginners. We want to make it easier for those who
want to pick up using \LaTeX ML in the future, by going through how to construct a minimal \LaTeX ML 
binding step by step.  \\

\large\paragraph  This document does not cover advanced topics related \LaTeX  ML, and thus if 
you are interested the general theories, probably you can have the manual hand in hand with this 
document to have a better comprehension between the theories and the application. In addition, I will refer you 
to the particular chapters in the manual, when needed.
 
\chapter{Your First \LaTeX ML Document Class} 
\section{Using LaTeXML}
We are going to talk about various aspects of \LaTeX ML, and then we will move onto the 
workflow of creating your first \LaTeX\ document class binding. In this tutorial, we use the command:

\begin{lstlisting}
latexml
\end{lstlisting}

 for converting \TeX \ document into *.sml 
One quick note in regards to \LaTeX ML installation, when you think you have finished installing \LaTeX ML, run
a simple conversion command within mockDoc.tex's directory. You should be able to see an XML interpretation of
mockDoc.tex either in a form a standard output or a newly-generated XML file. It is totally fine to see tons of mysterious 
error messages at this point, because we have created anything yet. Under some circumstances when your \LaTeX ML doesn't seem
to function, maybe you have overlooked the prerequisites such as libxml2 and libxslt. \\

\begin{tcolorbox}
\emph{For more information about how to use \LaTeX ML, please have a look at the \LaTeX MLmanual chapter 2:Using \LaTeX ML.}
\end{tcolorbox}

\section{How to Create A LaTeXML Binding}
The conversion from \TeX\  to XML is processed by \LaTeX ML. Basically \LaTeX ML maps the \TeX \ markups to
the XML markups, more specifically macros, primitives and constructors. That's why you are able to customize the conversion 
between \TeX\  and XML, in three ways, modifying the bindings used by \textbf{latexml}, adding your own bindings that has not been
implemented, and even creating your own \TeX \ style and \LaTeX\  binding which is exactly the goal of this tutorial.
\subsection{Things We need}
It probably would be a good idea to name every file after the same prefix which will make your life easier in the future. We need to have:\\ 

*.tex  as your source file, so you can have something to convert from. You can write down whatever you want and based on 
this .tex file, your other files will vary. Feel free to define your own macros into something unusual such that, even if you accidentally load the
\TeX\ binding in \LaTeX ML, the conversion will fail, ensuring all the conversion is done by our \LaTeX ML binding; \\ \\

 *.cls for \LaTeX pdf which essentially helps you to see what our .tex file is like in a pdf format, since pdflatex is unable to process the customized macros and things alike; \\ \\

*.cls.ltxml your \LaTeX ML binding, similar to the *.doc.cls you have for \LaTeX, but used for the conversion to other formats ; \\ \\

*.rnc The RelaxNG schema compact form, which defines the structure of your .tex, crucial for executing tasks like placing the tags correctly and auto closing the tags when needed; \\ \\

trang.jar(optional): \LaTeX ML  cannot process the compact form scheme, and therefore you need trang to convert your .rnc into .rng, unless you 
want to write your scheme in .rng from the first, albeit this approach is not recommended for lack of efficiency and diffculty of maintenance; \\ \\

After you have finished writing all the documents above, run \LaTeX ML, and then you should be able to see the converted xml of your .tex. In the following 
chapters I will explain how to construct your *.doc.ltxml and *.rnc and the dos and don'ts detail 

\subsection{Minimal \LaTeX ML}
Since \LaTeX binding is a Perl module, we need to initialize a binding file by add the followings in the beginning of the document class: \\
\lstinputlisting[firstline=1,lastline=4]{doc.cls.ltxml} 
At the end of \LaTeX ML don't forget to include \\
\begin{lstlisting}
1;
\end{lstlisting}
to make sure Perl work properly.
These are just the must dos, one has to follow. If you don't understand them, it is OK for now. Just use them and they work.
The meat of \LaTeX ML bindings come form the construction and constructors.\\ 

\begin{tcolorbox}
{\emph{It will be good idea to read the manual Chapter 4: Customization, before
your proceed and come back to see how the theories are implemented.}}
\end{tcolorbox}

Assuming you have read chapter 4 thoroughly, and get some feelings about how things work. Now you want to teach \LaTeX ML the new commands
you created in your .tex file. Let's look at an example below:\\

\begin{lstlisting}
DefConstructor(`\newline ',``<mock:break/>");
\end{lstlisting}

The reason why I use the break as an example  is because you might encounter problems dealing with break in \LaTeX ML. The two backlashes macro is preserved in pool package, that's why if you still use the regular newline break macro, your \LaTeX ML will have a malformed error. Renaming your newline macro in your .tex will solve the problem for you.

After you link your .tex file and .cls.ltxml file by changing your document class in your .tex into your \LaTeX ML binding name, in our case ``doc". \LaTeX ML will load your binding file, when it tries to do the conversion.\\ \\
You might be wondering how \LaTeX ML reads your biding. To put it in a simple way, during the conversion process, whenever \LaTeX ML encounters a macro or control sequence, it will look for its replacement in your binding and then put the replacement in xml. This is where things get a little tricky. How about the closing tag? Just like section macro, you declare where the section starts and were the next section starts, nevertheless, you never write now close section, so \LaTeX ML will never close the section tags? Yes and no. Indeed \LaTeX ML will have no clue of where to close the declared tags if we don't tell it when to do so. I solve the problem by using auto-\textgreater which has something to do with your scheme.

\subsection{RelaxNG Schema}
Schema is a crucial document that decides how the xml is constructed. When you are creating your own schema, it is a good idea to have your .tex document open side by side to make sure your scheme works well with your 
.tex file. \\ \\
One good approach to test this is to create your expected xml output according to your .tex and then validate the 
test xml with your scheme. You can easily accomplish this by using emacs nxml mode in which you have the freedom 
to write your expected xml, while validating your xml at the same time. If validation fails, you can see the error message instantly, such that you can debug your xml or schema accordingly.
\begin{tcolorbox}
Tutorial: \href{http://www.emacswiki.org/emacs/NxmlMode}{Emacs: Nxml Mode} 
\end{tcolorbox} 

In our mockDoc.rnc, you can easily see under a document, there can be either p or section and under a section there can be the possibilities of having a title followed by p or a title followed by a subsection. The reason for this is because in mockDoc.tex in the first section, there is no subsection but text directly but in the other sections, there are subsections. What I am trying to say is, in your schema you need to consider all kinds of possible hierarchy of your elements.

\begin{tcolorbox}
Before you write your expected xml and RelaxNG schema, having a look at the links below can be beneficial: \\ \\
I.\href{http://relaxng.org/compact-tutorial-20030326.html}{RelaxNG Syntax Tutorial}; \\ \\
II. \href{http://www.w3schools.com/xml/}{XML tutorialsl}. 
\end{tcolorbox} 

\paragraph{Some more improvements:} If you have followed what I said, very likely you still have many errors when you use \LaTeX ML to compile your files. Don't be frustrated by this, when I tried to make my first binding, this "HowTo" didn't exit at all. The success is within a reach. We only need to deal with two more things, namespace and putting spaces in your text. 

We have a default namespace in the schema and we need to declare the schema in the binding and associate the prefix with the namespace. That's an easy step. Then we come to the obscure command of putting spaces between two words. It is related to the architecture of \LaTeX ML, which is far beyond the scope of this tutorial. So you can just do what is in the doc.cls.ltxml.\\
\lstinputlisting[firstline=11,lastline=11]{doc.cls.ltxml} 
Now you should have a minimal setup of what is required for a \LaTeX ML biding. \\ \\ \\ \\ \\
Congratulations for being able to follow this tutorial to the end. After the processing the makefile, you should be able to see the generated XML in your current directory which hopefully should look something similar to your expected XML! \\

\textbf{Reference} \\
 I. Bruce R. Miller, November 21, 2014 \emph{\LaTeX ML The Manual} \\
 II. 2006 - 2012 Philipp Lehman, \emph{biblatex}
 % III. kwarc.bib

\end{document}