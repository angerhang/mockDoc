\documentclass{book}

\title{\textbf{How To Write A Minimal LaTeXML Binding}}
\author{Author: Hang Yuan, Jinbo Zhang \\ \\ \\ Supervisor: Michael Kohlhase}

\begin{document}
\maketitle

\tableofcontents

\chapter{Introduction}
\LaTeX has been widely used as the word processing tools among scholars, especially 
when one needs to use large quantities of mathematical representations. \LaTeX is also 
good for those who is meticulous about typographical quality of the documents.  However,
\LaTeX lacks of conversion tool to XML which Digital Library of Mathematical Functions use
for delivery. DLMF developed \LaTeX ML , trying to make a new typesetting system that allows
users to be able to focus more on the content, but not the style by providing extensive ways of 
customizations. In order to achieve this goal, building up the document class binding seems crucial, 
and yet \LaTeX ML seems fairly unfathomable for beginners. We want to make it easier for people who
want to pick up using \LaTeX ML in the future, by going through how to construct a minimal \LaTeX ML 
binding.  \\

Feel free to skip the first chapter, if you have a good understanding of how macro programming works.
Albeit, even so, we suggest you skim over the first chapter since the way that \LaTeX ML defines 
new constructor, primitives..... are similar to how they are defined in Tex, therefore understanding
Tex macro gives you an edge in making your own \LaTeX ML document class.
  
\chapter{Understanding TeX}
\section{Make Customized TeX file}
\section{Programming in Macro}
\chapter{Understanding LaTeXML} 
\section{Using LaTeXML}
\section{LaTeXML Binding}
\subsection{Minimal LaTeXML Structure}
\subsection{RelaxNG Schema}
\chapter{References}



\end{document}