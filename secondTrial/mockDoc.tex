\documentclass{doc}
\begin{document}
	\section{A brief introduction about Shelley}
		Percy Bysshe Shelley (4 August 1792 -- 8 July 1822) was one of the major English Romantic poets, and is regarded by some critics as amongst the finest lyric poets in the English language. 
	\section{Ode to the West Wind (partial)}
		\subsection{I}
			\paragraph{1.}
				O wild West Wind, thou breath of Autumn’s being, \\ 
				Thou, from whose unseen presence the leaves dead \\ 
				Are driven, like ghosts from an enchante fleeing,
			\paragraph{2.}
				Yellow, and black, and pale, and hectic red,\\ 
				Pestilence-stricken multitudes: O thou, \\	
				Who chariotest to their dark wintry bed
			\paragraph{3.}
				The wingèd seeds, where they lie cold and low,\\	
				Each like a corpse within its grave, until \\
				Thine azure sister of the Spring shall blow
			\paragraph{4.}
				Her clarion o’er the dreaming earth, and fill \\ (Driving sweet buds like flocks to feed in air) \\ With living hues and odours plain and hill:
			\paragraph{5.}
				Wild Spirit, which art moving everywhere; \\ Destroyer and Preserver; hear, O hear!
		\subsection{II}
			\paragraph{1.}
				Thou on whose stream, ‘mid the steep sky’s commotion, \\
				Loose clouds like Earth’s decaying leaves are shed, \\
				Shook from the tangled boughs of Heaven and Ocean,
			\paragraph{2.}
				Angels of rain and lightning: there are spread \\
				On the blue surface of thine airy surge, \\
				Like the bright hair uplifted from the head
			\paragraph{3.}
				Of some fierce Maenad, even from the dim verge \\
				Of the horizon to the zenith’s height, \\
				The locks of the approaching storm. Thou dirge
			\paragraph{4.}
				Of the dying year, to which this closing night \\
				Will be the dome of a vast sepulchre \\
				Vaulted with all thy congregated might
			\paragraph{5.}
				Of vapours, from whose solid atmosphere \\
				Black rain, and fire, and hail will burst: O hear!
\end{document}
%%% Local Variables:
%%% mode: latex
%%% TeX-master: t
%%% End:
